\documentclass{unswmaths}

\usepackage{unswshortcuts}

\begin{document}

\subject{Modules and Representation Theory}
\author{Edward McDonald}
\title{Assignment 2}
\studentno{3375335}


\setlength\parindent{0pt}

\newcommand{\Ft}{\mathbb{F}_3}
\newcommand{\isom}{\cong}

\unswtitle{}
    
\section*{Question 1}
Let $R = \Rl[x]$.
\begin{lemma}
    Let $K$ be the submodule of $R^2$ generated by
    \begin{equation*}
        \begin{pmatrix}x+x^2\\2x+x^2\end{pmatrix},\begin{pmatrix}2x+x^2\\4x+x^2\end{pmatrix}.
    \end{equation*} 
    Then
    \begin{equation*}
        R^2/K = \frac{\Rl[x]}{\langle x \rangle}\oplus\frac{\Rl[x]}{\langle x^2\rangle}
    \end{equation*}
\end{lemma}
\begin{proof}
    We consider the matrix
    \begin{equation*}
        \begin{pmatrix}
            x+x^2 & 2x+x^2\\
            2x+x^2 & 4x+x^2
        \end{pmatrix}
    \end{equation*}
    This is simply
    \begin{equation*}
        x\begin{pmatrix}
            1+x & 2+x\\
            2+x & 4+x
        \end{pmatrix}
    \end{equation*}
    Subtracting twice the first row from the second, this becomes
    \begin{equation*}
        x\begin{pmatrix}
            1+x & 2+x\\
            1 & 2
        \end{pmatrix}.
    \end{equation*}
    Now subtract $(1+x)$ times the second row from the first,
    \begin{equation*}
        x\begin{pmatrix}
            0 & -x\\
            1 & 2
        \end{pmatrix}.
    \end{equation*}
    Now subtract the twice the first column from the second, and we obtain
    \begin{equation*}
        \begin{pmatrix}
            0 & -x^2\\
            x & 0
        \end{pmatrix}.
    \end{equation*}
    Hence, the image of the matrix
    \begin{equation*}
    \begin{pmatrix}
        x & 0\\
        0 & -x^2
    \end{pmatrix}
    \end{equation*}
    generates a submodule $N$, such that
    \begin{equation*}
        R^2/K \isom R^2/N.
    \end{equation*}
    Hence,
    \begin{equation*}
        R^2/K \isom \frac{R^2}{Rx\oplus Rx^2} \isom R/\langle x\rangle \oplus R/\langle x^2\rangle.
    \end{equation*}
\end{proof}
    
\begin{proposition}
    Suppose that $p_1(x),p_2(x),p_3(x),p_4(x) \in R$. Suppose that $N$ is the submodule of $R^2$
    generated by
    \begin{equation*}
        \begin{pmatrix}
            p_1\\p_2
        \end{pmatrix},
        \begin{pmatrix}
            p_3\\p_4
        \end{pmatrix}
    \end{equation*}
    Let $d(x)$ be the determinant of 
    \begin{equation*}
        \begin{pmatrix}
            p_1(x) & p_3(x)\\
            p_2(x) & p_4(x)
        \end{pmatrix}
    \end{equation*}
    Then $R^2/N$ is infinite dimensional as a real vector space if and only if $d(x) = 0$. Otherwise,
    \begin{equation*}
        \dim(R^2/N) = \deg d(x).
    \end{equation*}
\end{proposition}
\begin{proof}
    There exist matrices $\Phi_l$ and $\Phi_r$ in $M_n(R)$ such that
    \begin{equation*}
        \Phi_l\begin{pmatrix}
            p_1(x) & p_3(x)\\
            p_2(x) & p_4(x)
        \end{pmatrix}\Phi_r = \begin{pmatrix}
            q_1(x) & 0\\
            0 & q_2(x)
        \end{pmatrix}
    \end{equation*}
    where $\det(\Phi_l) = \det(\Phi_r) = 1$ and $q_1|q_2$. So,
    \begin{equation*}
        d(x) = \det\begin{pmatrix}
            p_1(x) & p_3(x)\\
            p_2(x) & p_4(x)
        \end{pmatrix} = q_1(x)q_2(x)
    \end{equation*}
    and
    \begin{equation*}
        R^2/N \isom \frac{R^2}{Rq_1(x)\oplus Rq_2(x)} \isom R/\langle q_1(x)\rangle \oplus R/\langle q_2(x)\rangle.
    \end{equation*}
    
    If $d(x) = 0$, then $q_1(x) = 0$ or $q_2(x) = 0$. Hence, in the case
    $d(x) = 0$, $R^2/N \isom R\oplus R/\langle q_2(x)\rangle$ or $R^2/N \isom R/\langle q_1(x)\rangle \oplus R$.
    So $R^2/N$ has infinite real dimension if $d(x) = 0$ since $R$ is infinite dimensional.
    
    If $d(x) \neq 0$, then $q_1(x) \neq 0$ and $q_2(x) \neq 0$. In this case, 
    $R^2/N$ must be finite dimensional since $R/\langle q_1(x)\rangle$ and $R/\langle q_2(x)$
    are spanned by monomials $1,x,x^2,\ldots,x^n$ for $n\leq \deg(q_1)$ and $n\leq \deg(q_2)$ respectively.
    
    Hence $R^2/N$ is infinite dimensional if and only if $d(x) = 0$.     
    
    If $R/\langle q_1(x)\rangle$ is finite dimensional, then it has real dimension $\deg(q_1)$
    since if $f(x) \in R$, then $f(x) = q(x)q_1(x) + r(x)$, where $\deg(r) < \deg(q_1)$, so 
    $r$ is a linear combination of $1,x,x^2,\ldots,x^n$ for $n < \deg(q_1)$. Hence these
    are $\deg(q_1)$ linearly independent spanning elements of $R/\langle q_1(x)\rangle$. 
    
    Similarly, if $R/\langle q_2(x)\rangle$ is finite dimensional, then it has real
    dimension $\deg(q_2)$.
    
    Hence, if $R^2/N$ has finite real dimension, 
    \begin{equation*}
        \dim_\Rl(R^2/N) = \deg(q_1)\deg(q_2) = \deg(d).
    \end{equation*}
\end{proof}

\section*{Question 2}
In this question, we consider the algebra $A = \Ft G$ where $G = \langle \sigma\rangle$
is the cyclic group of order $4$. We use the isomorphism,
\begin{equation*}
    A\isom \frac{\Ft[x]}{\langle x^4-1\rangle}, \sigma\mapsto x
\end{equation*}
Note that $x^4-1 = (x-1)(x+1)(x^2+1)$. This is a decomposition into prime factors,
since $x-1$ and $x+1$ are degree $1$, therefore prime and if $x^2+1$ has a proper factor,
then it has a linear factor. If $x^2+1$ has a linear factor, it has a root over $\Ft$. However
for $x \in \Ft$, $x^2+1\neq 0$.

\begin{lemma}
    The maximal ideals of $A$ are exactly
    \begin{align*}
        &\langle \sigma-1\rangle\\
        &\langle \sigma+1\rangle\\
        &\langle \sigma^2+1\rangle.
    \end{align*}
\end{lemma}
\begin{proof}
    
\end{proof}

    
\end{document}
