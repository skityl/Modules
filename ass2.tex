\documentclass{unswmaths}

\usepackage{unswshortcuts}

\begin{document}

\subject{Modules and Representation Theory}
\author{Edward McDonald}
\title{Assignment 2}
\studentno{3375335}


\setlength\parindent{0pt}

\unswtitle{}
    
\section*{Question 1}
Let $R = \Rl[x]$.
\begin{lemma}
    Let $K$ be the submodule of $R^2$ generated by
    \begin{equation*}
        \begin{pmatrix}x+x^2\\2x+x^2\end{pmatrix},\begin{pmatrix}2x+x^2\\4x+x^2\end{pmatrix}.
    \end{equation*} 
    Then
    \begin{equation*}
        R^2/K = \frac{\Rl[x]}{\langle x \rangle}\oplus\frac{\Rl[x]}{\langle x^2\rangle}
    \end{equation*}
\end{lemma}
\begin{proof}
    
\end{proof}
    
\begin{proposition}
    Suppose that $p_1(x),p_2(x),p_3(x),p_4(x) \in R$. Suppose that $N$ is the submodule of $R^2$
    generated by
    \begin{equation*}
        \begin{pmatrix}
            p_1\\p_2
        \end{pmatrix},
        \begin{pmatrix}
            p_3\\p_4
        \end{pmatrix}
    \end{equation*}
    Let $d(x)$ be the determinant of 
    \begin{equation*}
        \begin{pmatrix}
            p_1(x) & p_3(x)\\
            p_2(x) & p_4(x)
        \end{pmatrix}
    \end{equation*}
    Then if $d = 0$, then $R^2/N$ is infinite dimensional as a real vector space. Otherwise,
    \begin{equation*}
        \dim(R^2/N) = \deg d(x).
    \end{equation*}
\end{proposition}

    
    
\end{document}