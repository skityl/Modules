\documentclass{unswmaths}

\usepackage{unswshortcuts}

\begin{document}

\subject{Modules and Representation Theory}
\author{Edward McDonald}
\title{Assignment 2}
\studentno{3375335}


\setlength\parindent{0pt}

\newcommand{\Ft}{{\mathbb{F}_3}}
\newcommand{\isom}{\cong}
\newcommand{\im}{{\operatorname{im}}}


\unswtitle{}
    
\section*{Question 1}
Let $R = \Rl[x]$.
\begin{lemma}
    Let $K$ be the submodule of $R^2$ generated by
    \begin{equation*}
        \begin{pmatrix}x+x^2\\2x+x^2\end{pmatrix},\begin{pmatrix}2x+x^2\\4x+x^2\end{pmatrix}.
    \end{equation*} 
    Then
    \begin{equation*}
        R^2/K = \frac{\Rl[x]}{\langle x \rangle}\oplus\frac{\Rl[x]}{\langle x^2\rangle}
    \end{equation*}
\end{lemma}
\begin{proof}
    We consider the matrix
    \begin{equation*}
        \begin{pmatrix}
            x+x^2 & 2x+x^2\\
            2x+x^2 & 4x+x^2
        \end{pmatrix}
    \end{equation*}
    This is simply
    \begin{equation*}
        x\begin{pmatrix}
            1+x & 2+x\\
            2+x & 4+x
        \end{pmatrix}
    \end{equation*}
    Subtracting twice the first row from the second, this becomes
    \begin{equation*}
        x\begin{pmatrix}
            1+x & 2+x\\
            1 & 2
        \end{pmatrix}.
    \end{equation*}
    Now subtract $(1+x)$ times the second row from the first,
    \begin{equation*}
        x\begin{pmatrix}
            0 & -x\\
            1 & 2
        \end{pmatrix}.
    \end{equation*}
    Now subtract the twice the first column from the second, and we obtain
    \begin{equation*}
        \begin{pmatrix}
            0 & -x^2\\
            x & 0
        \end{pmatrix}.
    \end{equation*}
    Hence, the image of the matrix
    \begin{equation*}
    \begin{pmatrix}
        x & 0\\
        0 & -x^2
    \end{pmatrix}
    \end{equation*}
    generates a submodule $N$, such that
    \begin{equation*}
        R^2/K \isom R^2/N.
    \end{equation*}
    Hence,
    \begin{equation*}
        R^2/K \isom \frac{R^2}{Rx\oplus Rx^2} \isom R/\langle x\rangle \oplus R/\langle x^2\rangle.
    \end{equation*}
\end{proof}
    
\begin{proposition}
    Suppose that $p_1(x),p_2(x),p_3(x),p_4(x) \in R$. Suppose that $N$ is the submodule of $R^2$
    generated by
    \begin{equation*}
        \begin{pmatrix}
            p_1\\p_2
        \end{pmatrix},
        \begin{pmatrix}
            p_3\\p_4
        \end{pmatrix}
    \end{equation*}
    Let $d(x)$ be the determinant of 
    \begin{equation*}
        \begin{pmatrix}
            p_1(x) & p_3(x)\\
            p_2(x) & p_4(x)
        \end{pmatrix}
    \end{equation*}
    Then $R^2/N$ is infinite dimensional as a real vector space if and only if $d(x) = 0$. Otherwise,
    \begin{equation*}
        \dim(R^2/N) = \deg d(x).
    \end{equation*}
\end{proposition}
\begin{proof}
    There exist matrices $\Phi_l$ and $\Phi_r$ in $M_n(R)$ such that
    \begin{equation*}
        \Phi_l\begin{pmatrix}
            p_1(x) & p_3(x)\\
            p_2(x) & p_4(x)
        \end{pmatrix}\Phi_r = \begin{pmatrix}
            q_1(x) & 0\\
            0 & q_2(x)
        \end{pmatrix}
    \end{equation*}
    where $\det(\Phi_l),\det(\Phi_r) \neq 0$ and $q_1|q_2$. So,
    \begin{equation*}
        d(x) = \det\begin{pmatrix}
            p_1(x) & p_3(x)\\
            p_2(x) & p_4(x)
        \end{pmatrix} = \det(\Phi_l)^{-1}\det(\Phi_r)^{-1}q_1(x)q_2(x)
    \end{equation*}
    and
    \begin{equation*}
        R^2/N \isom \frac{R^2}{Rq_1(x)\oplus Rq_2(x)} \isom R/\langle q_1(x)\rangle \oplus R/\langle q_2(x)\rangle.
    \end{equation*}
    
    If $d(x) = 0$, then $q_1(x) = 0$ or $q_2(x) = 0$. Hence, in the case
    $d(x) = 0$, $R^2/N \isom R\oplus R/\langle q_2(x)\rangle$ or $R^2/N \isom R/\langle q_1(x)\rangle \oplus R$.
    So $R^2/N$ has infinite real dimension if $d(x) = 0$ since $R$ is infinite dimensional.
    
    If $d(x) \neq 0$, then $q_1(x) \neq 0$ and $q_2(x) \neq 0$. In this case, 
    $R^2/N$ must be finite dimensional since $R/\langle q_1(x)\rangle$ and $R/\langle q_2(x)$
    are spanned by monomials $1,x,x^2,\ldots,x^n$ for $n\leq \deg(q_1)$ and $n\leq \deg(q_2)$ respectively.
    
    Hence $R^2/N$ is infinite dimensional if and only if $d(x) = 0$.     
    
    If $R/\langle q_1(x)\rangle$ is finite dimensional, then it has real dimension $\deg(q_1)$
    since if $f(x) \in R$, then $f(x) = q(x)q_1(x) + r(x)$, where $\deg(r) < \deg(q_1)$, so 
    $r$ is a linear combination of $1,x,x^2,\ldots,x^n$ for $n < \deg(q_1)$. Hence these
    are $\deg(q_1)$ linearly independent spanning elements of $R/\langle q_1(x)\rangle$. 
    
    Similarly, if $R/\langle q_2(x)\rangle$ is finite dimensional, then it has real
    dimension $\deg(q_2)$.
    
    Hence, if $R^2/N$ has finite real dimension, 
    \begin{equation*}
        \dim_\Rl(R^2/N) = \deg(q_1)\deg(q_2) = \deg(d).
    \end{equation*}
\end{proof}

\section*{Question 2}
In this question, we consider the algebra $A = \Ft G$ where $G = \langle \sigma\rangle$
is the cyclic group of order $4$. We use the isomorphism,
\begin{equation*}
    A\isom \frac{\Ft[x]}{\langle x^4-1\rangle}, \sigma\mapsto x
\end{equation*}
Note that $x^4-1 = (x-1)(x+1)(x^2+1)$. This is a decomposition into prime factors,
since $x-1$ and $x+1$ are degree $1$, therefore prime and if $x^2+1$ has a proper factor,
then it has a linear factor. If $x^2+1$ has a linear factor, it has a root over $\Ft$. However
for $x \in \Ft$, $x^2+1\neq 0$.

\begin{lemma}
    The maximal ideals of $A$ are exactly
    \begin{align*}
        &\langle \sigma-1\rangle\\
        &\langle \sigma+1\rangle\\
        &\langle \sigma^2+1\rangle.
    \end{align*}
\end{lemma}
\begin{proof}
    Ideals of $\Ft[x]/\langle x^4-1\rangle$ are of the form $\langle f(x)\rangle/\langle x^4-1\rangle$
    for some $f(x)|x^4-1$. An ideal $\langle f(x) \rangle/\langle x^4-1\rangle$ is maximal
    if and only if
    \begin{equation*}
        \frac{\Ft[x]/\langle x^4-1\rangle}{\langle f(x)\rangle/\langle x^4-1\rangle} \isom \frac{\Ft[x]}{\langle f(x)\rangle}
    \end{equation*}
    is a field. That is, $f(x)$ must be an irreducible divisor of $x^4-1$. Hence we
    have three choices for $f(x)$:
    \begin{align*}
        &x^2+1\\
        &x-1\\
        &x+1.
    \end{align*}
    So in $A$, this corresponds to $\sigma^2+1$, $\sigma-1$ or $\sigma+1$. Hence the required
    maximal ideals are $\langle \sigma^2+1\rangle,\langle\sigma-1\rangle,\langle \sigma+1\rangle$. 
\end{proof}

\begin{theorem}
\label{extWeddDecomp}
    $A$ has Wedderburn decomposition,
    \begin{equation*}
        A \isom \frac{\Ft[x]}{\langle x-1\rangle}\times\frac{\Ft[x]}{\langle x+1\rangle}\times\frac{\Ft[x]}{\langle x^2+1\rangle} \isom \Ft\times\Ft\times\mathbb{F}_9.
    \end{equation*}
\end{theorem}
\begin{proof}
    This follows from the chinese remainder theorem, since $A\isom \Ft[x]/\langle x^4-1\rangle$, and
    the ideals
    \begin{align*}
        &\langle x-1\rangle\\
        &\langle x+1\rangle\\
        &\langle x^2+1\rangle.
    \end{align*}
    generate $\Ft[x]$, are generated by coprime polynomials and have intersection $\langle x^4-1\rangle$, we have
    \begin{equation*}
        A \isom \frac{\Ft[x]}{\langle x^4-1\rangle} \isom \frac{\Ft[x]}{\langle x-1\rangle}\times \frac{\Ft[x]}{\langle x+1\rangle}\times\frac{\Ft[x]}{\langle x^2+1\rangle} \isom \Ft\times\Ft\times\mathbb{F}_9.
    \end{equation*}
\end{proof}

\begin{lemma}
\label{intWeddDecomp}
    The Wedderburn decomposition in theorem \ref{extWeddDecomp} corresponds to
    an isomorphism
    \begin{equation*}
        \frac{\Ft[x]}{\langle x-1\rangle}\times\frac{\Ft[x]}{\langle x+1\rangle}\times\frac{\Ft[x]}{\langle x^2+1\rangle}\isom \frac{\Ft[x]}{\langle x^4-1\rangle}
    \end{equation*}
    given by
    \begin{equation*}
        (a(x),b(x),c(x))\mapsto a(x)(x+1)(x^2+1)+b(x)(x-1)(x^2+1)+c(x)(x-1)(x+1).
    \end{equation*}
\end{lemma}
\begin{proof}
    The isomorphism
    \begin{equation*}
        \frac{\Ft[x]}{\langle x^4-1\rangle} \rightarrow \frac{\Ft[x]}{\langle x-1\rangle}\times\frac{\Ft[x]}{\langle x+1\rangle}\times\frac{\Ft[x]}{\langle x^2+1}
    \end{equation*}
    is given by
    \begin{equation*}
        f(x) \mapsto (f(x)+\langle x-1\rangle,f(x)+\langle x+1\rangle,f(x)+\langle x^2+1\rangle).
    \end{equation*}
    So we wish to find $e_1(x),e_2(x),e_3(x) \in \Ft[x]/\langle x^4-1\rangle$ such that $e_1(x)\mapsto (1,0,0)$,
    $e_2(x)\mapsto (0,1,0)$ and $e_3(x)\mapsto (0,0,1)$ under this isomorphism.
    
    So we require $e_1(x) \in \langle x+1\rangle\cap\langle x^2+1\rangle$, and 
    $e_1(x) + \langle x-1\rangle = 1+\langle x-1\rangle$. The only choice for $e_1(x)$
    is $(x+1)(x^2+1)$. 
    
    Similarly, $e_2(x) = (x-1)(x^2+1)$ and $e_3(x) = (x-1)(x+1)$. 
    
    Hence, the isomorphism
    \begin{equation*}
        \frac{\Ft[x]}{\langle x-1\rangle}\times\frac{\Ft[x]}{\langle x+1\rangle}\times\frac{\Ft[x]}{\langle x^2+1}\rightarrow \frac{\Ft[x]}{\langle x^4-1\rangle}
    \end{equation*}
    maps $(1,0,0)$ to $e_1(x)$, $(0,1,0)$ to $e_2(x)$ and $(0,0,1)$ to $e_3(x)$, and 
    so by $\Ft[x]$-linearity, the result follows.
\end{proof}

\begin{definition}
    Let $\rho:G \rightarrow \operatorname{GL}_3(\Ft)$ be the $\Ft$-linear representation
    of $G$ given by 
    \begin{equation*}
        \rho(\sigma) = \begin{pmatrix} 1 & 0 & 0\\ 
            0 & 0 & -11\\
            -1 & 1 & 0
        \end{pmatrix}
    \end{equation*}
    
    The corresponding $\Ft G$-module $V = \Ft^3$ is defined for $v \in V$ by
    \begin{equation*}
        \left(\sum_{g\in G} \alpha_g g\right)v = \sum_{g\in G}\alpha_g\rho(g)v
    \end{equation*} 
    
    
    See that
    \begin{equation*}
        \rho(\sigma)^2 = \rho(\sigma^2) = \begin{pmatrix}
            1 & 0 & 0\\
            1 & -1 & 0\\
            -1 & 0 & -1
        \end{pmatrix}   
    \end{equation*}
    Hence,
    \begin{align*}
        (\sigma^2-1)\begin{pmatrix} 1\\1\\0\end{pmatrix} &= ( \rho(\sigma)^2-I)\begin{pmatrix} 1\\1\\0\end{pmatrix}\\
        &=\begin{pmatrix}
            0 & 0 & 0\\
            1 & 1 & 0\\
            -1 & 0 & 1
        \end{pmatrix}
        \begin{pmatrix}
            1\\1\\0
        \end{pmatrix}\\
        &=
        \begin{pmatrix}
            0\\-1\\-1
        \end{pmatrix}
    \end{align*}
    
\end{definition}

\begin{proposition}
    If $\langle f(x)\rangle/\langle x^4-1\rangle$ is a maximal ideal of $\Ft[x]/\langle x^4-1\rangle$, then
    this corresponds to a maximal ideal $\langle f(\sigma)\rangle$ of $A$ and the corresponding
    isotypic component of $V$ is $\ker f(\rho(\sigma))$.
\end{proposition}
\begin{proof}
    The isotypic components of $V$ correspond to the images of left multiplication
    by the multiplicative identities in each Wedderburn component. In the notation
    of lemma $\ref{intWeddDecomp}$, 
    this means that the isotypic components are
    \begin{align*}
        &e_1(\rho(\sigma))V\\
        &e_2(\rho(\sigma))V\\
        &e_3(\rho(\sigma))V.
    \end{align*}
    
    To find the image of $e_1(\rho(\sigma))$, note that since
    \begin{equation*}
        V = e_1(\rho(\sigma))V+e_2(\rho(\sigma))V+e_3(\rho(\sigma))V
    \end{equation*}
    we may write
    \begin{align*}
        \im e_1(\rho(\sigma)) &= \ker e_2(\rho(\sigma))\cap\ker e_3(\rho(\sigma))\\
                              &= \ker (\rho(\sigma)-I)(\rho(\sigma)^2+I)\cap \ker (\rho(\sigma)-I)(\rho(\sigma)+I)
    \end{align*}
    Clearly $\ker \rho(\sigma)-I \subset \ker (\rho(\sigma)-I)(\rho(\sigma)^2+I)\cap \ker (\rho(\sigma)-I)(\rho(\sigma)+I)$,
    and the opposite inclusion holds because $I = \rho(\sigma)^2+I-\rho(\sigma)(\rho(\sigma)+I)$.
    
    Similarly,
    \begin{align*}
        e_1(\rho(\sigma))V &= \ker (\rho(\sigma)-I)\\
        e_2(\rho(\sigma))V &= \ker (\rho(\sigma)+I)\\
        e_3(\rho(\sigma))V &= \ker (\rho(\sigma)^2+I).
    \end{align*}    
\end{proof}
\begin{remark}
    The preceding result states that
    \begin{equation*}
        V = \ker (\rho(\sigma)-I)\oplus \ker (\rho(\sigma)+I)\oplus \ker (\rho(\sigma)^2+I)
    \end{equation*}
    An identical decomposition could be obtained by primary decomposition, since
    \begin{equation*}
        (\rho(\sigma)^2+I)(\rho(\sigma)+I)(\rho(\sigma)-I) = 0
    \end{equation*}
    and the polynomials
    \begin{align*}
        &(x+1)(x-1)\\
        &(x+1)(x^2+1)\\
        &(x-1)(x^2+1)
    \end{align*}
    are coprime.
\end{remark}

\begin{corollary}
    Hence the isotypic components of $V$ are
    \begin{equation*}
        \Ft\begin{pmatrix}
            -1\\1\\-1
        \end{pmatrix},\;\;\Ft\begin{pmatrix}0\\1\\0  \end{pmatrix}\oplus\Ft\begin{pmatrix} 0\\0\\1\end{pmatrix},\;\;0.
    \end{equation*}
\end{corollary}
\begin{proof}
    $\ker (\rho(\sigma)-I)$ is simply
    \begin{equation*}
        \ker\begin{pmatrix}
            0 & 0 & 0\\
            0 & -1 & -1\\
            -1 & 1 & -1
        \end{pmatrix} = 0.
    \end{equation*}
    
    Similarly, $\ker(\rho(\sigma)+I)$ is
    \begin{equation*}
        \ker\begin{pmatrix}
            -1 & 0 & 0\\
            0 & 1 & -1\\
            -1 & 1 & 1
        \end{pmatrix} = \Ft\begin{pmatrix}
                -1 \\1 \\-1
        \end{pmatrix}
    \end{equation*}
    And $\ker(\rho(\sigma)^2+I)$ is
    \begin{equation*}
        \ker\begin{pmatrix}
            -1 & 0 & 0\\
            1 & 0 & 0\\
            -1 & 0 & 0
        \end{pmatrix}
        = \Ft\begin{pmatrix}
            0\\0\\1
        \end{pmatrix}\oplus
        \begin{pmatrix}
            0\\1\\0
        \end{pmatrix}       
    \end{equation*}
\end{proof}

\section*{Question 3}
For this question, $G = \langle \sigma,\tau\;|\;\sigma^3=1,\tau^2=1,\tau\sigma = \sigma^{-1}\tau\rangle$,
and $A = \Ft G$.

\begin{theorem}
\label{1dimreps}
    There are two one dimensional representations of $G$ in $\Ft$, given by
    \begin{align*}
        &\sigma\mapsto 1,\;\tau\mapsto 1\\
        &\sigma\mapsto 1,\;\tau\mapsto -1\\
    \end{align*}
\end{theorem}
\begin{proof}
        If $f:G\rightarrow \Ft^\times = \{1,-1\}$ is a one dimensional representation, it
        must have $f(\sigma) = 1$, since $f(\sigma)^3 = f(\sigma)^3 = 1$. 
        Then we may choose $f(\tau) = 1$ or $f(\tau) = -1$, and this uniquely
        determines the representation by the universal property of free groups.
\end{proof}

\begin{definition}
    The representation $\rho:G\rightarrow \operatorname{GL}_2(\Ft)$ is given by
    \begin{equation*}
        \rho(\sigma) = \begin{pmatrix}
            0 & -1\\
            1 & -1
        \end{pmatrix},\;\;
        \rho(\tau) = \begin{pmatrix}
            1 & -1\\
            0 & -1
        \end{pmatrix}
    \end{equation*}
    $V = \Ft^2$ is the corresponding $A$-module.
\end{definition}

\begin{proposition}
\label{notSemisimple}
    $V$ is not semisimple.
\end{proposition}
\begin{proof}
    Consider the subspace $\Ft v$, where
    \begin{equation*}
        v = \begin{pmatrix}
            1\\-1
        \end{pmatrix}.
    \end{equation*}
    $\Ft v$ is an $A$-submodule of $V$, since $\rho(\sigma)v = v$ and 
    $\rho(\tau)v = -v$.
    
    If $V$ is semisimple, then $\Ft v$ must be a direct summand of $V$. Hence there
    is another submodule $\Ft u$ for some $u$ not parallel to $v$ such that
    $V = \Ft v+\Ft u$.
    
    However if $\Ft u$ is a submodule of $V$, then $u$ must be an eigenvector
    of $\rho(\sigma)$. 
    
    However, the characteristic polynomial of $\rho(\sigma)$ is 
    $cp_{\rho(\sigma)}(\lambda) = (\lambda-1)^2$, and $\ker \rho(\sigma)-I = \Ft v$.
    
    Hence $\rho(\sigma)$ has no eigenvectors other then $v$, and so $\Ft v$ cannot
    be a direct summand of $V$. Hence $V$ is not semisimple.
\end{proof}

\begin{proposition}
    A composition series for $V$ is 
    \begin{equation*}
        0 < \Ft v < V
    \end{equation*} 
    where $v = (1,-1)^\top$ as in proposition \ref{notSemisimple}. The composition
    factors are 
    \begin{equation*}
        \Ft v, V/\Ft v
    \end{equation*}
    which are isomorphic as $\Ft$ modules to $\Ft$, but as $A$ modules
    they correspond to the representations in theorem \ref{1dimreps}.
\end{proposition}
\begin{proof}
    We have already shown in proposition \ref{notSemisimple} that $\Ft v$ is an
    $A$-submodule of $V$. So it is required to show that $0 < \Ft v < V$ is a composition
    series, by showing that the composition factors are simple.
    
    Clearly
    \begin{equation*}
        \Ft v/0, V/\Ft v
    \end{equation*}
    are simple, since they are one dimensional as vector spaces over $\Ft$, hence
    can have no nontrivial $A$-submodules. 
    
    These composition factors are one dimensional $\Ft$-vector spaces,
    and $A$-modules, so correspond to one dimensional representations of $G$. 
   
    See that since $\rho(\sigma)v = v$ and $\rho(\tau)v = -v$, the first composition factor
    $\Ft v/0$ corresponds to the nontrivial representation in theorem \ref{1dimreps}.
    
    For $u+\Ft v \in V/\Ft v$, the action of $g$ on $u+\Ft v$ is given by
    $\rho(g)(u)+\Ft v$. This is well defined since $v$ is an eigenvector
    of $\rho(\sigma)$ and $\rho(\tau)$, so $\Ft v$ is invariant under the action of $G$.

    Elements of $V/\Ft v$ can be described as
    \begin{equation*}
        \alpha \begin{pmatrix}
            1\\0
        \end{pmatrix}
        +\Ft v
    \end{equation*} 
    for some $\alpha \in \Ft$
    ince if $u+\Ft v$ is any coset of $\Ft v$, then we may find $\alpha \in \Ft$
    such that $u+\Ft v = \alpha(1,0)^\top +\Ft v$.
    
    Hence, for any $\alpha(1,0)^\top+\Ft v \in V/\Ft v$, we may compute
    \begin{align*}
        \rho(\sigma)(\alpha(1,0)^\top+\Ft v) &= \alpha (1,0)+\Ft v\\
        \rho(\tau)(\alpha(1,0)^\top +\Ft v) &= \alpha(1,0)+\Ft v
    \end{align*}
    So this corresponds to the trivial representation in theorem \ref{1dimreps}.  
    
\end{proof}
    
\end{document}
